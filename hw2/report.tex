\documentclass[12pt, a4paper]{article} % book, report, article, letter, slides
                                       % letterpaper/a4paper, 10pt/11pt/12pt, twocolumn/twoside/landscape/draft

%%%%%%%%%%%%%%%% PACKAGES %%%%%%%%%%%%%%%%%%%%%

\usepackage[utf8]{inputenc} % encoding

\usepackage[english]{babel} % use special characters and also translates some elements within the document.

\usepackage{amsmath}        % Math
\usepackage{amsthm}         % Math, \newtheorem, \proof, etc
\usepackage{amssymb}        % Math, extended collection
\usepackage{bm}             % $\bm{D + C}$
\newtheorem{theorem}{Theorem}[section]     % \begin{theorem}\label{t:label}  \end{theorem}<Paste>
\newtheorem{corollary}{Corollary}[theorem]
\newtheorem{lemma}[theorem]{Lemma}
\newenvironment{claim}[1]{\par\noindent\underline{Claim:}\space#1}{}
\newenvironment{claimproof}[1]{\par\noindent\underline{Proof:}\space#1}{\hfill $\blacksquare$}

\usepackage{hyperref}       % Hyperlinks \url{url} or \href{url}{name}

\usepackage{parskip}        % \par starts on left (not idented)

\usepackage{abstract}       % Abstract

\usepackage{tocbibind}      % Adds the bibliography to the table of contents (automatically)

\usepackage{graphicx}       % Images
\graphicspath{ {./images/} }

\usepackage[vlined,ruled]{algorithm2e} % pseudo-code

% \usepackage[document]{ragged2e}  % Left-aligned (whole document)
% \begin{...} ... \end{...}   flushleft, flushright, center

%%%%%%%%%%%%%%%% CODE %%%%%%%%%%%%%%%%%%%%%

\usepackage{minted}         % Code listing
% \mint{html}|<h2>Something <b>here</b></h2>|
% \inputminted{octave}{BitXorMatrix.m}

%\begin{listing}[H]
  %\begin{minted}[xleftmargin=20pt,linenos,bgcolor=codegray]{haskell}
  %\end{minted}
  %\caption{Example of a listing.}
  %\label{lst:example} % You can reference it by \ref{lst:example}
%\end{listing}

\newcommand{\code}[1]{\texttt{#1}} % Define \code{foo.hs} environment

%%%%%%%%%%%%%%%% COLOURS %%%%%%%%%%%%%%%%%%%%%

\usepackage{xcolor}         % Colours \definecolor, \color{codegray}
\definecolor{codegray}{rgb}{0.9, 0.9, 0.9}
% \color{codegray} ... ...
% \textcolor{red}{easily}

%%%%%%%%%%%%%%%% CONFIG %%%%%%%%%%%%%%%%%%%%%

\renewcommand{\absnamepos}{flushleft}
\setlength{\absleftindent}{0pt}
\setlength{\absrightindent}{0pt}

%%%%%%%%%%%%%%%% GLOSSARIES & ACRONYMS %%%%%%%%%%%%%%%%%%%%%

%\usepackage{glossaries}

%\makeglossaries % before entries

%\newglossaryentry{latex}{
    %name=latex,
    %description={Is a mark up language specially suited
    %for scientific documents}
%}

% Referene to a glossary \gls{latex}
% Print glossaries \printglossaries

\usepackage[acronym]{glossaries} %

% \acrshort{name}
% \acrfull{name}
\newacronym{kcol}{$k$-COL}{$k$-coloring problem}
\newacronym{scol}{SEARCH-$k$-COL}{Search $k$-coloring problem}
\newacronym{2col}{$2$-COL}{$2$-coloring problem}
\newacronym{e2sat}{$E2$-SAT}{Exactly 2-SAT}

%%%%%%%%%%%%%%%% HEADER %%%%%%%%%%%%%%%%%%%%%

\usepackage{fancyhdr}
\pagestyle{fancy}
\fancyhf{}
\rhead{Arnau Abella \- MIRI}
\lhead{Computational Complexity}
\rfoot{Page \thepage}

%%%%%%%%%%%%%%%% TITLE %%%%%%%%%%%%%%%%%%%%%

\title{%
  Computational Complexity: Homework 2
}
\author{%
  Arnau Abella \\
  \large{Universitat Polit\`ecnica de Catalunya}
}
\date{\today}

%%%%%%%%%%%%%%%% DOCUMENT %%%%%%%%%%%%%%%%%%%%%

\begin{document}

\maketitle

%%%%%%%%%% Exercise 1

\section{Self-reducibility}

\textbf{Exercise: Self-reducibility} Show that if the decision version \acrshort{kcol} of the \acrfull{kcol} can be solved in polynomial time, then its search version \acrshort{scol} can also be solved in polynomial time.

We are going to describe an algorithm that can solve \acrshort{scol} in polynomial time.

\begin{algorithm}[H]
  \SetAlgoNoLine
  \SetKwInOut{Input}{Input}
  \SetKwInOut{Output}{Output}
  \Input{Given the graph $G = \langle V, E \rangle$ and $k$ colors.}
  \Output{A solution $a := a_1,\dots,a_{|V|}$ of \acrshort{scol}, $a_i = \{1..k\}$ \\ or $UNSAT$.}
  \BlankLine
  \For(\tcp*[f]{$|V|$ times}){$i \leftarrow 1$ \KwTo $|V|$}{
    \For(\tcp*[f]{$|K|$ times}){$j \leftarrow 1$ \KwTo $|k|$}{
      \tcp{Each assignment is a copy of the vector $a$}
      \tcp{At most $|v|^{|k|}$ copies}
      $a := (a_1,\dots,a_{i-1}, a_i = j)$;
    }
    \If{$i = |V|$}{
      Test if $x := (x_1 = a_1, \dots, x_n = a_{|V|})$ is a valid assignment using \acrshort{kcol}; \tcp{p(n)-time}
      If it is a valid assignment, then \emph{halt} and return $a$;
     }
  }
  return $UNSAT$;
  \caption{Na\"ive \acrshort{scol} Algorithm}
  \label{scol}
\end{algorithm}

\newpage

The total cost of the algorithm \ref{scol} is $\mathcal{O}(|k|^{|V|} \cdot n^c)$ where $|k| \leq n$ and $c > 0$.

We found a polynomial time algorithm that solves \acrshort{scol} in polynomial time.

%%%%%%%%%% Exercise 2

\section{Zero-error}

\textbf{Exercise: Zero-error} Show that if \acrshort{kcol} is in BPP, then it is also in ZPP. In other words, show that if there is a probabilistic polynomial time algorithm that decides whether a graph has a valid \acrshort{kcol}, and does so with bounded error, then there is also a probabilistic polynomial time algorithm that does that with zero error (in the sense of the definition of ZPP).

Since \acrshort{kcol} is NP-complete, we can generalize the problem to

\[
  NP \subseteq BPP \implies NP = ZPP
\]

The basic inclusions are the following:

\begin{align*}
  P \subseteq ZPP \subseteq RP   &\subseteq NP   \\
  P \subseteq ZPP \subseteq coRP &\subseteq coNP \\
                              RP &\subseteq BPP  \\
                            coRP &\subseteq BPP
\end{align*}

$P$ is contained in $ZPP = RP \cap coRP$, which is contained in $BPP$. Moreover, $RP$ is contained in $NP$ and $coRP$ is contained in $coNP$. \textbf{But we don't know how $NP$ and $coNP$ relate to $BPP$}.

It follows from $NP \subseteq BPP$, that $NP = RP$. That means that $ coRP = coNP$ and $ZPP = NP \cap coNP $. So basically in this case, we have $P$, which contains $NP$ and $coNP$, and those are both contained in $BPP$.

\textbf{Assuming} $\bm{NP = coNP}$, which is an \textit{open problem}, we can show that \acrshort{kcol} is in $ZPP$.


%%%%%%%%%% Exercise 3

\newpage

\section{Logarithmic space}

\textbf{Exercise: Logarithmic space} Show that $2$-COL is in P. Show that it is even in NL. [Since it is known that 3-COL is NP-complete, make sure that you are using something special about $2$-COL in your proof, and state what.]

First, let's review what we know so far.

\begin{theorem}\label{teo}
  \acrshort{e2sat} $\in$ $NL$
\end{theorem}

\begin{proof}
  Given a \acrshort{2col} F, we construct the \underline{implication graph $G(F)$} of F:
  \begin{itemize}
    \item One vertex for every literal
    \item Two edge $\neg l \to l'$, $\neg l' \to l$ for every clause
  \end{itemize}

  \begin{claim}
    If $G(F)$ contains a cycle that contains a variable and it's negation, then $F$ is $UNSAT$.
  \end{claim}
  \begin{claim}
    (contrapositive) If $G(F)$ does not contains a cycle that contains a variable and it's negation, then $F$ is $SAT$.
  \end{claim}
\end{proof}

Therefore, if we can give an algorithm to \textit{reduce} \acrshort{kcol} to \acrshort{e2sat} in polynomial time, we can use the theorem \ref{teo} to solve it in polynomial polynomial time.

The algorithm is the following:

  \begin{itemize}
    \item There will be one clause for each edge, and one literal for each vertex.
    \item Each clause will contain exactly two literals that represent the edge $\vec{uv}$, and each literal will codify the color of the vertex $i$ as there are only two possible colors.
  \end{itemize}

\end{document}

\documentclass[12pt, a4paper]{article} % book, report, article, letter, slides
                                       % letterpaper/a4paper, 10pt/11pt/12pt, twocolumn/twoside/landscape/draft

%%%%%%%%%%%%%%%% PACKAGES %%%%%%%%%%%%%%%%%%%%%

\usepackage[utf8]{inputenc} % encoding

\usepackage[english]{babel} % use special characters and also translates some elements within the document.

\usepackage{amsmath}        % Math
\usepackage{amsthm}         % Math, \newtheorem, \proof, etc
\usepackage{amssymb}        % Math, extended collection
\usepackage{bm}             % $\bm{D + C}$
\newtheorem{theorem}{Theorem}[section]     % \begin{theorem}\label{t:label}  \end{theorem}<Paste>
\newtheorem{corollary}{Corollary}[theorem]
\newtheorem{lemma}[theorem]{Lemma}
\newenvironment{claim}[1]{\par\noindent\underline{Claim:}\space#1}{}
\newenvironment{claimproof}[1]{\par\noindent\underline{Proof:}\space#1}{\hfill $\blacksquare$}

\usepackage{hyperref}       % Hyperlinks \url{url} or \href{url}{name}

\usepackage{parskip}        % \par starts on left (not idented)

\usepackage{abstract}       % Abstract

\usepackage{tocbibind}      % Adds the bibliography to the table of contents (automatically)

\usepackage{graphicx}       % Images
\graphicspath{ {./images/} }

\usepackage[vlined,ruled]{algorithm2e} % pseudo-code

% \usepackage[document]{ragged2e}  % Left-aligned (whole document)
% \begin{...} ... \end{...}   flushleft, flushright, center

%%%%%%%%%%%%%%%% CODE %%%%%%%%%%%%%%%%%%%%%

\usepackage{minted}         % Code listing
% \mint{html}|<h2>Something <b>here</b></h2>|
% \inputminted{octave}{BitXorMatrix.m}

%\begin{listing}[H]
  %\begin{minted}[xleftmargin=20pt,linenos,bgcolor=codegray]{haskell}
  %\end{minted}
  %\caption{Example of a listing.}
  %\label{lst:example} % You can reference it by \ref{lst:example}
%\end{listing}

\newcommand{\code}[1]{\texttt{#1}} % Define \code{foo.hs} environment

%%%%%%%%%%%%%%%% COLOURS %%%%%%%%%%%%%%%%%%%%%

\usepackage{xcolor}         % Colours \definecolor, \color{codegray}
\definecolor{codegray}{rgb}{0.9, 0.9, 0.9}
% \color{codegray} ... ...
% \textcolor{red}{easily}

%%%%%%%%%%%%%%%% CONFIG %%%%%%%%%%%%%%%%%%%%%

\renewcommand{\absnamepos}{flushleft}
\setlength{\absleftindent}{0pt}
\setlength{\absrightindent}{0pt}

%%%%%%%%%%%%%%%% GLOSSARIES & ACRONYMS %%%%%%%%%%%%%%%%%%%%%

%\usepackage{glossaries}

%\makeglossaries % before entries

%\newglossaryentry{latex}{
    %name=latex,
    %description={Is a mark up language specially suited
    %for scientific documents}
%}

% Referene to a glossary \gls{latex}
% Print glossaries \printglossaries

\usepackage[acronym]{glossaries} %

% \acrshort{name}
% \acrfull{name}
% \newacronym{kcol}{$k$-COL}{$k$-coloring problem}

%%%%%%%%%%%%%%% COMMANDS

\newcommand{\sumip}{\sum_i^p}
\newcommand{\sumipp}{\sum_{i-1}^p}
\newcommand{\prodip}{\prod_i^p}
\newcommand{\prodipp}{\prod_{i-1}^p}

%%%%%%%%%%%%%%%% HEADER %%%%%%%%%%%%%%%%%%%%%

\usepackage{fancyhdr}
\pagestyle{fancy}
\fancyhf{}
\rhead{Arnau Abella \- MIRI}
\lhead{Computational Complexity}
\rfoot{Page \thepage}

%%%%%%%%%%%%%%%% TITLE %%%%%%%%%%%%%%%%%%%%%

\title{%
  Computational Complexity: Homework 3
}
\author{%
  Arnau Abella \\
  \large{Universitat Polit\`ecnica de Catalunya}
}
\date{\today}

%%%%%%%%%%%%%%%% DOCUMENT %%%%%%%%%%%%%%%%%%%%%

\begin{document}

\maketitle

%%%%%%%%%% Exercise 1

\section{Oracles in BPP}

\textbf{Exercise: Oracles in BPP} \quad Prove that ${BPP}^{BPP} = BPP$.

\begin{proof}
  In order to prove equality we need to proof both inclusions:

  \begin{align*}
    BPP^{BPP} &\subseteq BPP \\
    BPP &\subseteq BBP^{BPP}
  \end{align*}

  Let $L \in BPP^{BPP}$ be the class of languages decided by a $BPP$ machine $M$ with access to an oracle $O \in BPP$. The oracle $O$ decides the class of languages in $BPP$ in constant time (instead of polynomial time) using a $PTM$ $M'$ with a probability of error $E'$ e.g. $E' \leq 1/3$.

  For every $x \in \{0,1\}^*$, $M$ applies the same algorithm used in the \textit{error reduction} method but making at most $p(|x|)$ independent calls to the oracle machine $M'$, where $p : \mathbb{N} \to \mathbb{N}$. The overall probability of error by $M$ deciding $L$ is $E < E'$. Hence, $\mathbf{BPP \subseteq BPP^{BPP}}$.

  In the previous paragraph we saw that the probability of error of $M' \in BPP^{BPP}$ is $E$ which is smaller than the probability of error of $M \in BPP$. Let $M$ be in $BPP$, by \textit{error reduction} using $m \geq p(|x|)$ repitions, we can achieve a smaller probability of error then $E$. Therefore, $\mathbf{BPP^{BPP} \subseteq BPP}$.

\end{proof}

%%%%%%%%%% Exercise 2

\section{Co-classes and the hierarchy}

\textbf{Exercise: Co-classes and the hierarchy} \quad Prove that if $\sum_i^p = \prod_i^p$ for some $i \geq 1$, then the polynomial-time hierarchy collapses to its $i$-th level; i.e., $\sum_i^p = \prod_i^p = PH$

\begin{proof}

  Let $L \in \sumip$. By definition, there is a polynomial-time Turing machine $M$ and a polynomial $q$ such that

  \begin{equation}\label{eq:one}
    x \in L \iff \exists u_1 \in \{0,1\}^{q(|x|)} \forall u_2 \in \{0,1\}^{q(|x|)} \cdots Q_i u_i \in \{0,1\}^{q(|x|)} M(x, u_1, \cdots, u_i) = 1
  \end{equation}

  where $Q_i$ is $\exists / \forall$ alternatively. Define the language $L'$ as follows:

  \begin{equation}
    \langle x, u_1 \rangle \in L' \iff \forall u_2 \in \{0,1\}^{q(|x|)} \cdots Q_i u_i \in \{0,1\}^{q(|x|)} M(x, u_1, \cdots, u_i) = 1
  \end{equation}

  Clearly, $L' \in \prodipp$ and so under assumption $L'$ is in $\sumipp$. This implies that:

  \begin{equation}\label{eq:two}
    \begin{split}
      \langle x, u_1 \rangle \in L' \iff & \exists v_1 \in \{0,1\}^{q(|x|)} \forall v_2 \in \{0,1\}^{q(|x|)}  \cdots Q_{i-1} v_{i-1} \in \{0,1\}^{q(|x|)} \\
      &M(x, v_1, \cdots, v_{i-1}) = 1
    \end{split}
  \end{equation}

  Plugging $L'$ in \ref{eq:one}, we get:

  \begin{equation}
    x \in L \iff \exists u_1 \in \{0,1\}^{q(|x|)} (x,u_1) \in L'
  \end{equation}

  Replacing $L'$ by the definition of \ref{eq:two}:

  \begin{equation}
    \begin{split}
      x \in L \iff & \exists(u_1, v_1) \in \{0,1\}^{q(|x|)} \forall v_2 \in \{0,1\}^{q(|x|)} \cdots Q_{i-1} v_{i-1} \in \{0,1\}^{q(|x|)} \\
      & M(x, v_1, \cdots, v_{i-1}) = 1
    \end{split}
  \end{equation}

  But this means $L \in \sumipp$ and hence $\sumip = \sumipp$, then the hierarchy collapses.

\end{proof}

%%%%%%%%%% Exercise 3

\section{Circuits and the hierarchy}

\textbf{Exercise: Circuits and the hierarchy} \quad Prove that if $NP \subseteq P/poly$, then $NP^{NP} \subseteq P/poly$. It is also possible to prove that $NP \subseteq P/poly$ implies $\sum_2^p = \prod_2^p$. This is called the "Karp-Lipton Theorem".

\begin{proof}

  $P/poly$ is the class of languages that are decidable by polynomial-sized circuit families i.e. $P/poly = \cup_cSIZE(n^c)$, $c > 0$.

  If $NP \subseteq P/poly$, then any $NP$ problem can be solved by a polynomial-sized circuit. An efficient translation would be reducing the problem to $SAT$ and direclty encoding the Boolean formula into a Boolean circuit.

  Let's prove that any language in $NP^{NP}$ can be decided by the equivalent polynomial-sized circuit. An $NP^{NP}$ problem is the kind of problems that can be decided in polynomial time by a $NTM$ $M$ with access to an oracle $O \in NP$. Under the assumption that $NP \subseteq P/poly$, we can solve calls to the oracle $O$ using polynomial-sized circuits (notice, we are no longer talking about time but size of a circuit). So, the resulting circuit is the one that where the nodes which made calls to the oracle $O$ in the original problem, are replaced by the $n^d$-sized circuit that decides problems in $NP$. The overall size of the circuit $O(n^c n^d)$ $c,d > 0$ is still polynomial.

\end{proof}

  \subsubsection{Karp-Lipton Theorem: $NP \subseteq P/poly \implies PH = \sum_2^p$}

  \begin{proof}

    To show $PH = \sum_2^p$, it is enough showing that $\sum_2^p = \prod_2^p$ (as we proved in the second exercise).
  In particular, it suffices to show $\prod_2^p \subseteq \sum_2^p$ i.e. $\sum_2^p$ contains the $\prod_2^p$-complete language $\prod_2SAT$:

  \begin{equation}\label{eq:6}
    \prod_2SAT: \forall u \in \{0,1\}^n \ \exists v \in \{0,1\}^n \ \varphi(u,v) = 1
  \end{equation}

  If $NP \subseteq P/poly$, then there exists a Boolean circuit family $\{C_n\}_{n \in \mathbb{N}}$ such that for every Boolean formula $\varphi$ and $u \in \{0,1\}^n$, $C_n(\varphi, u) = 1$ iff there exists $v \in \{0,1\}^n$ s.t. $\varphi(u,v) = 1$. Notice $C_n$ solves the decision problem for $SAT$.

  Recalling from previous lectures, if $P = NP$, we can give a polynomial time algorithm that transform any decision problem to the equivalent search problem.
  Let's express this algorithm as a circuit. There exist a $q(n)$-sized circuit family $\{C'_n\}_{n \in \mathbb{N}}$, where $q$ is a polynomial function, such that for every formula $\varphi$ and $u \in \{0,1\}^n$, if there is a string $v \in \{0,1\}^n$ s.t. $\varphi(u,v) = 1$, then $C'_n(\varphi, u)$ output such string $v$. The assumption of $NP \subseteq P/poly$ implies the existence of such circuits.

  The main idea of Karp-Lipton is that such a circuit can be \textit{"guessed"} using $\exists$ quantifications.
  Since the circuit outputs a satisfying assignment if one exists, this answer can be checked directly.
  Formaly, since $C'_n$ can be described using $10q(n)^2$ bits, if (\ref{eq:6}) holds then the following quantified formula is true:

  \begin{equation}\label{eq:7}
     \forall u \in \{0,1\}^n \exists w \in \{0,1\}^{10q(n)^2} \ \text{s.t. $w$ describes a circuit C' and} \ \varphi(u, C'(\varphi, u))) = 1
  \end{equation}

  Furthermore, if (\ref{eq:6}) is false, then for some $u$, no $v$ exists such that $\varphi(u,v) = 1$ and hence (\ref{eq:7}) is false as well.
  Thus (\ref{eq:6}) holds if and only if (\ref{eq:7}) holds.
  Finally, since evaluating a ciruit can be done deterministically in polynomial time, the truth of (\ref{eq:7}) can be verified in $\sum_2^p$.

  \end{proof}

\end{document}

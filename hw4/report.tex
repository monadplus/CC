\documentclass[12pt, a4paper]{article} % book, report, article, letter, slides
                                       % letterpaper/a4paper, 10pt/11pt/12pt, twocolumn/twoside/landscape/draft

%%%%%%%%%%%%%%%% PACKAGES %%%%%%%%%%%%%%%%%%%%%

\usepackage[utf8]{inputenc} % encoding

\usepackage[english]{babel} % use special characters and also translates some elements within the document.

\usepackage{amsmath}        % Math
\usepackage{amsthm}         % Math, \newtheorem, \proof, etc
\usepackage{amssymb}        % Math, extended collection
\usepackage{amsfonts}       % Math, Natural, Integers and so
\usepackage{bm}             % $\bm{D + C}$
\newtheorem{theorem}{Theorem}[section]     % \begin{theorem}\label{t:label}  \end{theorem}<Paste>
\newtheorem{corollary}{Corollary}[theorem]
\newtheorem{lemma}[theorem]{Lemma}
\theoremstyle{definition}
\newtheorem{definition}{Definition}[section]
\newenvironment{claim}[1]{\par\noindent\underline{Claim:}\space#1}{}
\newenvironment{claimproof}[1]{\par\noindent\underline{Proof:}\space#1}{\hfill $\blacksquare$}

\usepackage{hyperref}       % Hyperlinks \url{url} or \href{url}{name}

\usepackage{parskip}        % \par starts on left (not idented)

\usepackage{abstract}       % Abstract

\usepackage{tocbibind}      % Adds the bibliography to the table of contents (automatically)

\usepackage{graphicx}       % Images
\graphicspath{{./images/}}

\usepackage[vlined,ruled]{algorithm2e} % pseudo-code

% \usepackage[document]{ragged2e}  % Left-aligned (whole document)
% \begin{...} ... \end{...}   flushleft, flushright, center

%%%%%%%%%%%%%%%% CODE %%%%%%%%%%%%%%%%%%%%%

\usepackage{minted}         % Code listing
% \mint{html}|<h2>Something <b>here</b></h2>|
% \inputminted{octave}{BitXorMatrix.m}

%\begin{listing}[H]
  %\begin{minted}[xleftmargin=20pt,linenos,bgcolor=codegray]{haskell}
  %\end{minted}
  %\caption{Example of a listing.}
  %\label{lst:example} % You can reference it by \ref{lst:example}
%\end{listing}

\newcommand{\code}[1]{\texttt{#1}} % Define \code{foo.hs} environment

%%%%%%%%%%%%%%%% COLOURS %%%%%%%%%%%%%%%%%%%%%

\usepackage{xcolor}         % Colours \definecolor, \color{codegray}
\definecolor{codegray}{rgb}{0.9, 0.9, 0.9}
% \color{codegray} ... ...
% \textcolor{red}{easily}

%%%%%%%%%%%%%%%% CONFIG %%%%%%%%%%%%%%%%%%%%%

\renewcommand{\absnamepos}{flushleft}
\setlength{\absleftindent}{0pt}
\setlength{\absrightindent}{0pt}

%%%%%%%%%%%%%%%% GLOSSARIES & ACRONYMS %%%%%%%%%%%%%%%%%%%%%

%\usepackage{glossaries}

%\makeglossaries % before entries

%\newglossaryentry{latex}{
    %name=latex,
    %description={Is a mark up language specially suited
    %for scientific documents}
%}

% Referene to a glossary \gls{latex}
% Print glossaries \printglossaries

\usepackage[acronym]{glossaries} %

% \acrshort{name}
% \acrfull{name}
% \newacronym{kcol}{$k$-COL}{$k$-coloring problem}

%%%%%%%%%%%%%%% COMMANDS

\newcommand{\Z}{\mathbb{Z}}

%%%%%%%%%%%%%%%% HEADER %%%%%%%%%%%%%%%%%%%%%

\usepackage{fancyhdr}
\pagestyle{fancy}
\fancyhf{}
\rhead{Arnau Abella \- MIRI}
\lhead{Computational Complexity}
\rfoot{Page \thepage}

%%%%%%%%%%%%%%%% TITLE %%%%%%%%%%%%%%%%%%%%%

\title{%
  Computational Complexity: Homework 4
}
\author{%
  Arnau Abella \\
  \large{Universitat Polit\`ecnica de Catalunya}
}
\date{\today}

%%%%%%%%%%%%%%%% DOCUMENT %%%%%%%%%%%%%%%%%%%%%

\begin{document}

\maketitle

%%%%%%%%%% Exercise 1

\section{More on hash functions}

\textbf{Exercise: More on hash functions} \quad Let $p$ be a prime number, let $\Z_p$ denote the field of arithmetic mod $p$. For each $a,b,c \in \Z_p$, let ${P_{a,b,c,} : \Z_p \to \Z_p}$ be the quadratic polynomial defined by $P_{a,b,c}(x) = ax^2 + bx + c \ mod \ p$. Show that the family of functions ${ P_{a,b,c}: a,b,c \in \Z_p}$ satisfies the following 3-universal property: for every three distinct $x_1, x_2, x_3 \in \Z_p$ and every $y_1, y_2, y_3 \in \Z_p$ we have

\begin{equation}\label{eq:one}
  \underset{a,b,c \in_R \Z_p}{Pr} [ P_{a,b,c}(x_i) = y_i \ for \ i = 1,2,3] = \frac{1}{p^3}
\end{equation}

The notation "$a,b,c, \in_R \Z_p$" under "$Pr$" means that $a,b$ and $c$ are chosen uniformly and independently at random in $\Z_p$.

\begin{proof}

  Let $x_1 \neq x_2, x_1 \neq x_3, x_1 \neq x_3 \in \Z_p$ and $y_1, y_2, y_3 \in \Z_p$. Then we will show that $P_{a,b,c}$ is completely determined by the choice of $a,b,c \in \Z_p$ and the number of possible triplets is exactly $1/p^3$. This can be achieved by expressing the linear equation of the quadratic polynomials defined by $P_{a,b,c}$ in matrix form and applying the inverse (there is one because $p$ is prime and $x_1, x_2, x_3 \in \Z_p$) to the left hand-side:

\begin{align*}
  \underset{a,b,c \in_R \Z_p}{Pr} & [ P_{a,b,c}(x_1) = y_1 \land P_{a,b,c}(x_2) = y_2 \land P_{a,b,c}(x_3) = y_3]\\
  & = Pr \begin{bmatrix}A\begin{bmatrix}a\\b\\c\end{bmatrix} = \begin{bmatrix}y_1\\y_2\\y_3\end{bmatrix}\end{bmatrix} \\
  & = Pr \begin{bmatrix}A^{-1}A\begin{bmatrix}a\\b\\c\end{bmatrix} = A^{-1}\begin{bmatrix}y_1\\y_2\\y_3\end{bmatrix}\end{bmatrix}\\
  & = Pr \begin{bmatrix}\begin{bmatrix}a\\b\\c\end{bmatrix} = A^{-1}\begin{bmatrix}y_1\\y_2\\y_3\end{bmatrix}\end{bmatrix}\\
  & = Pr \begin{bmatrix}\begin{bmatrix}a\\b\\c\end{bmatrix} = \frac{1}{|A|}Adj(A)\begin{bmatrix}y_1\\y_2\\y_3\end{bmatrix}\end{bmatrix}\\
  & = 1/p^3
\end{align*}

where $A = \begin{bmatrix} x_1^2 & x_1 & 1\\ x_2^2 & x_2 & 1\\ x_3^2 & x_3 & 1 \end{bmatrix}$

\end{proof}

%%%%%%%%%% Exercise 2

\newpage

\section{More on hasing for estimating sizes of sets}

\textbf{More on hasing for estimating sizes of sets} \quad Let $m$ and $k$ be positive integers and let $U = U_{m,k}$ be a 2-universal family of hash functions from $m$ bits to $k$ bits. For any fixed set $S \subseteq \{0,1\}^m$ and a randomly chosen $h \in U$, let $I(S,h)$ be the indicator random variable for the event that $h$ has collision on $S$: "there exists two distinct $x,y \in S$ with $h(x) = h(y)$". Prove the following:

\begin{enumerate}
  \item If $|S| > 2^{k}$, then $Pr_{h \in_r U} [I(S,h) = 1] = 1$ \label{prop:one}
  \item If $|S| < 2^{k/2}$ , then $Pr_{h \in_r U} [I(S,h) = 1] \leq 1/2$ \label{prop:two}
\end{enumerate}

Recall that the notation "$h \in_r U$" under "$Pr$" means that $h$ is chosen uniformly at random in $U$.

\begin{definition}
  Let $U$ be a universe with $|U| \geq n$ and let $V = \{0,1,\cdots, n- 1\}$. A family of hash functions $\mathcal{H}$ from $U$ to $V$ is said to be \textbf{k-universal} if, for any elements $x_1, x_2, \cdots, x_k$ and for a hash function $h$ chosen uniformly at random from $\mathcal{H}$, we have

  \[ Pr(h(x_1) = h(x_2) = \cdots = h(x_k)) \leq \frac{1}{n^{k-1}} \]
\end{definition}

%%% Property 1.
\textbf{Property 1.} \quad If $|S| > 2^{k}$, then $Pr_{h \in_r U} [I(S,h) = 1] = 1$

\begin{proof}


  Given $|S| >  2^k$, the total number of possible combinations of pairs is $\binom{2^k}{2}$.
  Suppose that $V$ is at most of size $|S|^2$.
  Therefore, the overall probability of the random variable $I = 1$ is

  \begin{align*}
    Pr_{h \in_r U} [I(S,h) = 1] &\leq \binom{2^k}{2}\frac{1}{2^{2k}}\\
                                &\leq \frac{2^k(2^k - 1)}{2 \cdot 2^k \cdot 2^k}\\
                                &\leq \frac{2^k - 1}{2 \cdot 2^k}\\
                                &\approx 1 \ (\text{when} \ k > 1)
  \end{align*}

\end{proof}

%%% Property 2.
\textbf{Property 2.} \quad If $|S| < 2^{k/2}$ , then $Pr_{h \in_r U} [I(S,h) = 1] \leq 1/2$

\begin{proof}


  Given $|S| <  2^{k/2}$, suppose that the size of $V$ is at least $|S|^2$.
  Therefore, the overall probability of the random variable $I = 1$ is

  \begin{align*}
    Pr_{h \in_r U} [I(S,h) = 1] &\leq \binom{2^{k/2}}{2}\frac{1}{2^k}\\
                                &\leq \frac{2^{k/2}(2^{k/2} - 1)}{2 \cdot 2^k}\\
                                &\leq \frac{2^{k/2} - 1}{2 \cdot 2^{k/2}}\\
                                &< \frac{1}{2}
  \end{align*}


\end{proof}

Notice that we fixed the size of $V$ to be at least $|S|^2$ in both cases. In order to make the probability of the first property be $= 1$ (instead of $\approx 1$), we must fix the size of $V$ to be at most $|S|$.

%%%%%%%%%% Exercise 3

\section{Approximate counting up to a square}

\textbf{Approximate counting up to a square} \quad Use the result of the previous exercise to show that for any $\#P$ function $f$ there exists a \textit{deterministic} polynomial time algorithm that, given an $n$-bit input $x$ and access to an $NP^{NP}$-oracle, outputs a number $t$ satisfying $t \leq f(x) \leq t^2$

\begin{proof}

  This proof is based on the following Stanford lecture's note \cite{lecture}. I adapted the proof to use an $NP^{NP}$-oracle instead of a $BPP^{NP}$-oracle by relaxing the $O(1)$-approximation to a $poly$-approximation.

  We first make some observations so that we can reduce the proof.

  \begin{itemize}
    \item It is enought to prove that theorem for $\#3SAT$. If we have an approximation algorithm for $\#3SAT$, we can extend it to any $\#A$ in $\#P$ using the \textit{parsimonious reduction} from $\#A$ to $\#3SAT$.
    \item It is enought to give a polynomial time $O(1)$-approximation for $\#3SAT$.

          Suppose we have an algorithm $C$ and a constant $c$ such that

          \begin{equation}\label{eq:2}
            \frac{1}{c}\#3SAT(\varphi) \leq C(\varphi) \leq c \cdot \#3SAT(\varphi)
          \end{equation}

          Given $\varphi$, we can construct $\varphi^k = \varphi_1 \land \varphi_2 \land \cdots \land \varphi_k$ where each $\phi_1$ is a copy of $\phi$ constructed using fresh variables. If $\varphi$ has $t$ satisfying assignments, $\varphi^k$ has $t^k$ satisfying assignments. Then, given $\varphi^k$ to the algorithm we get

          \begin{align*}
            \frac{1}{c}t^k \leq &C(\varphi^k) \leq ct^k \\
            \frac{1^{1/k}}{c}t^k \leq &C(\varphi^k)^{1/k} \leq c^{1/k}t
          \end{align*}

          If $c$ is a constant and $k = O(\frac{1}{\epsilon}), c^{1/k} = 1 + \epsilon$.

        \item For a formula $\varphi$ that has O(1) satisfying assignments, $\#3SAT(\varphi)$ can be found in $P^{NP}$. This can be done by iteratively asking the oracle the questions of the form: "Are there $k$ assignments satisfying this formula?" Notice that these are $NP$ questions, because the algorithm can guess these $k$ assignments and check them.
  \end{itemize}

  Suposse that we had available an approximate comparison procedure a-comp with the following properties:

  \begin{itemize}
    \item If $\#3SAT(\varphi) \geq 2^{k+1}$ then $a- comp(\varphi, k) = YES$ with high probability.
    \item If $\#3SAT(\varphi) \leq 2^{k}$ then $a- comp(\varphi, k) = NO$ with high probability.
  \end{itemize}

  Given $a-comp$, we can construct an algorithm that $2$-approximates $\#3SAT$ as described below:

  \begin{algorithm}[H]
    \SetAlgoNoLine
    \SetKwInOut{Input}{Input}
    \SetKwInOut{Output}{Output}
    \Input{$\varphi$}
    \For{$t = 0,1,\cdots, n + 1$}{
      $a-comp(\varphi, t)$
    }
    \uIf{$a-comp$ outputs $NO$ from the first time}{
      The value is either 0 or 1 and the answer can be checked by one more query to the $NP$-oracle. \\
      Query to the oracle and output an exact value.
    }
    \Else{
      Suppose that it outputs $YES$ for $t = 1,\cdots,i-1$ and $NO$ for $t=1$ \\
      Output $2^i$
    }
    \caption{2-approximation of $\#3SAT$}
    \label{2-approx}
  \end{algorithm}

  \newpage

  We show that this algorithm approximates $\#3SAT$ within a factor of $2$.
  If $a-comp$ answers $NO$ from the first time, the algorithm outputs the right answer because it checks for
  the answer explicitly. Now suppose $a-comp$ says $YES$ for all $t = 1,2,\cdots, i - 1$ ans says $NO$ for $t=i$.
  Since $a-comp(\varphi, i-1)$ outputs $YES$, $\#3SAT(\varphi) \geq 2^{i-1}$, and also since $a-comp(\varphi, 2^i)$ outputs $NO$, $\#3SAT < 2{i+1}$. The algorithm outputs $a = 2^i$. Hence,

  \begin{equation}\label{eq:3}
    \frac{1}{2}a \leq \#3SAT(\varphi) < 2 \cdot a
  \end{equation}

  and the algorithm outputs the correct answer with in a factor of 2.

  It is enough to give a $NP^{NP}$ implementation of the $a-comp$ procedure.

  The procedure and its analysis is similar to the Valiant-Vazirani reduction: for a given formula $\phi$ we pick a hash function $h$ from a $2$-universal family of hash functions, and look at the number of assignments $x$ that satisfy $h$ and such $h(x) = 0$.

  In the Valiant-Vazirani reduction, we proved that if $S$ is a set of size approximately equal to the size of the range of $h()$, then with constant probability, exactly one element of $S$ is mapped to $h()$ into 0. Now we use the result from exercise 2. From this, $a-comp$ can be constructed as follows.

  \begin{algorithm}[H]
    \SetAlgoNoLine
    \SetKwInOut{Input}{Input}
    \SetKwInOut{Output}{Output}
    \Input{$\varphi$, k}
    \uIf{$k \leq 5$}{
      Check exactly whether $\#3SAT(\varphi) \ge 2^k$.
    }
    \Else{
      Pick $h$ from the set of $2$-universal hash functions $h: \{0,1\}^n \to \{0,1\}^m$, where $m = k - 5$.
      Answer $YES$ iff there are more than $n$ assignments $a$ to $\varphi$ that $a$ satisfies $\varphi$ and $h(a)=0$.
    }
    \caption{$a-comp$}
    \label{2-approx}
  \end{algorithm}

  Notice that the test at the last step can be done with one access to an oracle to $NP$. We need to show that the algorithm is in $NP^{NP}$. Let $S \subseteq \{0,1\}^n$ be the set of satisfying assignments for $\varphi$. There are 2 cases:

  \begin{itemize}
    \item If $|S| \geq 2^{k}$, by \ref{prop:one} we have a high probability of success.
    \item If $|S| \leq 2^{k/2}$, by \ref{prop:two} we have $\leq 1/2$ of success, but using the error-reduction technique from the lectures, we can amplify the probability of success by $n$-calls to $a-comp$ and taking the majority answer.
  \end{itemize}


\end{proof}

\begin{thebibliography}{1}
  \bibitem{lecture} CS254 Lecture 8, \url{https://lucatrevisan.wordpress.com/2010/05/06/cs254-lecture-8-approximate-counting/}
\end{thebibliography}

\end{document}

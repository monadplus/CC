\documentclass[12pt, a4paper]{article} % book, report, article, letter, slides
                                       % letterpaper/a4paper, 10pt/11pt/12pt, twocolumn/twoside/landscape/draft

%%%%%%%%%%%%%%%% PACKAGES %%%%%%%%%%%%%%%%%%%%%

\usepackage[utf8]{inputenc} % encoding

\usepackage[english]{babel} % use special characters and also translates some elements within the document.

\usepackage{amsmath}        % Math
\usepackage{amsthm}         % Math, \newtheorem, \proof, etc
\usepackage{amssymb}        % Math, extended collection
\usepackage{amsfonts}       % Math, Natural, Integers and so
\usepackage{bm}             % $\bm{D + C}$
\newtheorem{theorem}{Theorem}[section]     % \begin{theorem}\label{t:label}  \end{theorem}<Paste>
\newtheorem{corollary}{Corollary}[theorem]
\newtheorem{lemma}[theorem]{Lemma}
\newenvironment{claim}[1]{\par\noindent\underline{Claim:}\space#1}{}
\newenvironment{claimproof}[1]{\par\noindent\underline{Proof:}\space#1}{\hfill $\blacksquare$}

\usepackage{hyperref}       % Hyperlinks \url{url} or \href{url}{name}

\usepackage{parskip}        % \par starts on left (not idented)

\usepackage{abstract}       % Abstract

\usepackage{tocbibind}      % Adds the bibliography to the table of contents (automatically)

\usepackage{graphicx}       % Images
\graphicspath{ {./images/} }

\usepackage[vlined,ruled]{algorithm2e} % pseudo-code

% \usepackage[document]{ragged2e}  % Left-aligned (whole document)
% \begin{...} ... \end{...}   flushleft, flushright, center

%%%%%%%%%%%%%%%% CODE %%%%%%%%%%%%%%%%%%%%%

\usepackage{minted}         % Code listing
% \mint{html}|<h2>Something <b>here</b></h2>|
% \inputminted{octave}{BitXorMatrix.m}

%\begin{listing}[H]
  %\begin{minted}[xleftmargin=20pt,linenos,bgcolor=codegray]{haskell}
  %\end{minted}
  %\caption{Example of a listing.}
  %\label{lst:example} % You can reference it by \ref{lst:example}
%\end{listing}

\newcommand{\code}[1]{\texttt{#1}} % Define \code{foo.hs} environment

%%%%%%%%%%%%%%%% COLOURS %%%%%%%%%%%%%%%%%%%%%

\usepackage{xcolor}         % Colours \definecolor, \color{codegray}
\definecolor{codegray}{rgb}{0.9, 0.9, 0.9}
% \color{codegray} ... ...
% \textcolor{red}{easily}

%%%%%%%%%%%%%%%% CONFIG %%%%%%%%%%%%%%%%%%%%%

\renewcommand{\absnamepos}{flushleft}
\setlength{\absleftindent}{0pt}
\setlength{\absrightindent}{0pt}

%%%%%%%%%%%%%%%% GLOSSARIES & ACRONYMS %%%%%%%%%%%%%%%%%%%%%

%\usepackage{glossaries}

%\makeglossaries % before entries

%\newglossaryentry{latex}{
    %name=latex,
    %description={Is a mark up language specially suited
    %for scientific documents}
%}

% Referene to a glossary \gls{latex}
% Print glossaries \printglossaries

\usepackage[acronym]{glossaries} %

% \acrshort{name}
% \acrfull{name}
% \newacronym{kcol}{$k$-COL}{$k$-coloring problem}

%%%%%%%%%%%%%%% COMMANDS

\newcommand{\Z}{\mathbb{Z}}

%%%%%%%%%%%%%%%% HEADER %%%%%%%%%%%%%%%%%%%%%

\usepackage{fancyhdr}
\pagestyle{fancy}
\fancyhf{}
\rhead{Arnau Abella \- MIRI}
\lhead{Computational Complexity}
\rfoot{Page \thepage}

%%%%%%%%%%%%%%%% TITLE %%%%%%%%%%%%%%%%%%%%%

\title{%
  Computational Complexity: Homework 4
}
\author{%
  Arnau Abella \\
  \large{Universitat Polit\`ecnica de Catalunya}
}
\date{\today}

%%%%%%%%%%%%%%%% DOCUMENT %%%%%%%%%%%%%%%%%%%%%

\begin{document}

\maketitle

%%%%%%%%%% Exercise 1

\section{More on hash functions}

\textbf{Exercise: More on hash functions} \quad Let $p$ be a prime number, let $\Z_p$ denote the field of arithmetic mod $p$. For each $a,b,c \in \Z_p$, let ${P_{a,b,c,} : \Z_p \to \Z_p}$ be the quadratic polynomial defined by $P_{a,b,c}(x) = ax^2 + bx + c \ mod \ p$. Show that the family of functions ${ P_{a,b,c}: a,b,c \in \Z_p}$ satisfies the following 3-universal property: for every three distinct $x_1, x_2, x_3 \in \Z_p$ and every $y_1, y_2, y_3 \in \Z_p$ we have

\begin{equation}\label{eq:one}
  \underset{a,b,c \in_R \Z_p}{Pr} [ P_{a,b,c}(x) = y_i \ for \ i = 1,2,3] = \frac{1}{p^3}
\end{equation}

The notation "$a,b,c, \in_R \Z_p$" under "$Pr$" means that $a,b$ and $c$ are chosen uniformly and independently at random in $\Z_p$.


\begin{proof}

  TODO

\end{proof}

%%%%%%%%%% Exercise 2

\section{More on hasing for estimating sizes of sets}

\textbf{More on hasing for estimating sizes of sets} \quad Let $m$ and $k$ be positive integers and let $U = U_{m,k}$ be a 2-universal family of hash functions from $m$ bits to $k$ bits. For any fixed set $S \subseteq \{0,1\}^m$ and a randomly chosen $h \in U$, let $I(S,h)$ be the indicator random variable for the event that $h$ has collision on $S$: "there exists two distinct $x,y \in S$ with $h(x) = h(y)$". Prove the following:

\begin{enumerate}
  \item If $|S| > 2^{2k}$, then $Pr_{h \in_r U} [I(S,h) = 1] = 1$
  \item If $|S| < 2^{k}$ , then $Pr_{h \in_r U} [I(S,h) = 1] \leq 1 - 2^{-k}$
\end{enumerate}

Recall that the notation "$h \in_r U$" under "$Pr$" means that $h$ is chosen uniformly at random in $U$.

\begin{proof}

  TODO

\end{proof}

%%%%%%%%%% Exercise 3

\section{Approximate counting up to a square}

\textbf{Approximate counting up to a square} \quad Use the result of the previous exercise to show that for any $\#P$ function $f$ there exists a \textit{deterministic} polynomial time algorithm that, given an $n$-bit input $x$ and access to an $NP^{NP}$-oracle, outputs a number $t$ satisfying $t \leq f(x) \leq t^2$

\begin{proof}

  TODO

\end{proof}

\end{document}
